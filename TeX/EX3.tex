\documentclass[12pt,letterpaper,twoside]{article}

\usepackage{cme212}

\begin{document}

{\centering \textbf{Paper Exercise 3\\ Due Friday, February 24th, in class \\}}
\vspace*{-8pt}\noindent\rule{\linewidth}{1pt}

\paragraph{(0)} Our graph is undirected, but we could use it to build a
\emph{directed} graph or \emph{digraph}. A digraph's edges are \emph{ordered}
pairs of nodes $(n_i, n_j)$, rather than unordered pairs $\{n_i, n_j\}$; all
operations look like their undirected counterparts and have similar complexity.

Here's one possible, and interesting, digraph representation.
\begin{cpp}
template <typename V> 
class Digraph { 
  ... 
 private:
  Graph<V> g_;
};
\end{cpp}
Explain the representation by giving an abstraction function and representation invariant. Note that \texttt{Digraph} is not \texttt{Graph}'s friend and can only use \texttt{Graph}'s public interface.


\paragraph{(1)} The \texttt{vector<T>::resize} operation extends a vector with
\emph{initialized} elements. (For example, after \texttt{vector<int> v;
  v.resize(20)}, \texttt{v} contains 20 zeroes.) This is almost always what we
want, but it can be useful to extend a vector with \emph{uninitialized
  memory}. Here's an example: uninitialized memory helps build a \emph{sparse
  vector}, in which elements are initialized only when first referenced. We
assume a special \texttt{garbage\_vector} type whose \texttt{resize} method does
not initialize new memory.

\begin{cpp}
template <typename T> 
class sparse_vector {
  garbage_vector<size_t> position_;
  vector<size_t> check_;
  vector<T> value_;
 public:
  /** Construct a sparse vector with all elements equal to T(). */
  sparse_vector() {
  }
  /** Return a reference to the element at position @a i. */
  T& operator[](size_t i) {
    if (i >= position_.size())
      position_.resize(i + 1);   // Adds uninitialized memory
    if (position_[i] >= check_.size() || check_[position_[i]] != i) {
      // This is the first reference to the element at position i
      position_[i] = check_.size();
      check_.push_back(i);
      value_.push_back(T());
    }
    return value_[position_[i]];
  }
};
\end{cpp}

An abstract \texttt{sparse\_vector} value is an \textbf{infinite} vector.

Write an abstraction function and representation invariant for
\texttt{sparse\_vector}.


\end{document}
