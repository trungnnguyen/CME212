\documentclass[12pt,letterpaper,twoside]{article}

\usepackage{cme212}

\begin{document}

{\centering \textbf{Paper Exercise 1 -- {\tt lower\_bound}\\ Due Friday, January 27, in class} \par}
\vspace*{-8pt}\noindent\rule{\linewidth}{1pt}
\vspace{-3em}

\paragraph{(0)} Consider the following specification and implementation for a {\tt lower\_bound} function:
\begin{cpp}
/** Find the first element in a sorted array that is not less than a value.
 * @param[in] a         Sorted array to search.
 * @param[in] low,high  Search in the index range [@a low, @a high).
 * @param[in] v         Value to search for.
 * @return  An index into array @a a or @a high.
 *
 * @tparam T Type of the elements. 
 *   T is comparable with 'bool operator<(T,T)'.
 *   For all values @a a[i] and @a v, operator< defines a strict weak ordering:
 *    -- Irreflexive: !operator<(x,x)
 *    -- Asymmetric:   operator<(x,y) => !operator<(y,x)
 *    -- Transitive:   operator<(x,y) & operator<(y,z) => operator<(x,z)
 *
 * @pre  0 <= @a low <= @a high <= Size of @a a.
 * @pre  For all i,j with @a low <= i < j < @a high, 
 *         @a a[i] <= @a a[j].
 *
 * @post For all i,j with @a low <= i < result <= j < @a high, 
 *         @a a[i] < @a v and @a a[j] >= @a v.
 *
 * Performs O(log(@a high - @a low)) operations.
 */
template <typename T>
int lower_bound(const T* a, int low, int high, const T& v) {
   while (low < high) {
      int mid = low + (high-low) / 2;
      if (a[mid] < v)
         low = mid+1;
      else
         high = mid; 
   }
   return low;
}
\end{cpp}
This specification is not {\em consistent}: it only requires {\tt operator<} but
uses {\tt <=} and {\tt >=} in its pre- and post-conditions.

More subtly, but also more importantly, it is also not {\em minimal}: there are
alternative (weaker) preconditions that would allow this {\em same} {\tt
  lower\_bound} to be used by a {\em wider range of inputs} -- inputs that the
current specification excludes. (Hint: Does the array have to be sorted? Does
\texttt{operator<} have to impose a strict weak ordering?).

Provide a more consistent and minimal specification.

\newpage


\paragraph{(1)} In class, we abstracted {\tt lower\_bound} to take a comparator
function object rather than always rely on {\tt operator<}. This is also
demonstrated in {\tt Examples/func\_params.cpp} in your code repository. In
fact, it's possible to define an interface to {\tt lower\_bound} that takes
\emph{only} a function object, rather than a value to search for and a comparator!
Write out a complete doxygen specification comment for this {\tt
  lower\_bound}. The implementation is not required. Pay special attention to
the signature and requirements of the function object.


\paragraph{(2)} Why might the C++ standard library have implemented the version with item and comparator, rather than the version with only a predicate?


\paragraph{(3)} Implement your {\tt lower\_bound} from (0) with your {\tt lower\_bound} from (1).

\end{document}
